\documentclass{article}
\usepackage[utf8]{inputenc}
\usepackage{amsthm,amsmath,amsfonts,amssymb}

\usepackage[margin=0.5in]{geometry}

\newcommand{\I}[1]{\mathcal{I}(#1)}
\newcommand{\V}[1]{\mathcal{V}(#1)}
\newcommand{\kn}{k[x_1,x_2,\ldots,x_n]}
\newcommand{\Rn}{R[x_1,x_2,\ldots,x_n]}
\newcommand{\Af}{{\Bbb A}^n}
\newcommand{\Ak}{{\Bbb A}_k^n}
\newcommand{\Akn}[1]{{\Bbb A}_k^#1}
\newcommand{\ARn}[1]{{\Bbb A}_{\Bbb R}^#1}
\newcommand{\An}[1]{{\Bbb A}^#1}
\newcommand{\fr}[1]{\mathfrak #1}
\newcommand{\C}{\mathbb C}
\newcommand{\Z}{\mathbb Z}
\newcommand{\Zn}[1]{\mathbb Z_{#1}}
\newcommand{\Q}{\mathbb Q}
\newcommand{\R}{\mathbb R}
\newcommand{\chn}[1]{C_\bullet(#1)}
\newcommand{\bkt}[1]{\langle #1 \rangle}

\newcommand{\ab}{\text{ab}}

\DeclareMathOperator{\im}{im}
\DeclareMathOperator{\Aut}{Aut}

\newcommand{\fix}[1]{\text{Fix}(#1)}

\theoremstyle{definition}
\newtheorem*{lem}{Lemma}
\newtheorem*{thm}{Theorem}
\newtheorem*{cor}{Corollary}
\newtheorem*{defn}{Definition}

\title{. . .}
\author{Soham}
\date{July 2016}

\begin{document}
\maketitle
 
\section{Tensor products of characters}
\begin{thm}
 The product of any two characters of the same finite group $G$ is again a
 character of $G$
\end{thm}
\begin{proof}
  The representation is $\rho = \rho_1\otimes\rho_2$. Taking its trace, we get a
  block-diagonal matrix whose character is
  $$\chi = \chi_1\chi_2.$$
\end{proof}

Let's compute the character table of $S_4$. The conjugacy classes are:

There is a natural character that sends a conjugacy class to the number of axes
in $\C^4$ which it preserves:

\begin{tabular}{l|lllll}
 $\chi_1$ &$1$  &$1$  &1  &1  &1 \\
 $\chi_2$ &1  &-1  &1  &-1   &1 \\
 $\chi_3$ &3  &1  &0   &-1 &-1 \\
 $\chi_4$ &  &  & & & \\ 
 $\chi_5$ &  &  & & &
\end{tabular}

How about squaring $\chi_3$? We get a character that contains a copy of $\chi_1$

$$9,1,0,1,1$$

and subtracting that off, and then doing the same with a copy of $\chi_3$ gives

$$5,-1,-1,1,1$$

But $\bkt{\chi,\chi} = \frac1{24}(25,6,8,6,3)=2$, so it's not irreducible. So it
must be equal to $\chi_4+\chi_5$. Note that, wlog, $\chi_4=\chi_2\chi_3$. We
finally get

\begin{tabular}{llllll}
 $\chi_1$ &1  &1  &1  &1  &1 \\
$\chi_2$  &1  &-1  &1  &-1   &1 \\
$\chi_3$  &3  &1  &0   &-1 &-1 \\
$\chi_4$  &3  &-1  &0  &1 &-1 \\ 
$\chi_5$  &2  &0  &-1  &0 &2
\end{tabular}

\section{Induced representations}
For any subgroup $H$ of a group $G$, and any rep $\rho:H\to GL(V)$, we can
define a representation of $G$, called the induced representation.

\begin{enumerate}
\item Find a set of representations for the left cosets of $H$ in $G$., i.e.,
  decompose $G$ as
  $$G=eH\cup a_2 H\cup\cdots a_q H,\,q=|G|/|H|$$
\item The space for our new rep has dimension $dq$. Our new representation
  $\sigma$ sends $g$ to a $q\times q$ block matrix, where each block is $d\times
  d$. We put $\rho({a_i^{-1}}g a_j)$ in the $i,j$ block if ${a_i^{-1}}ga_j\in H$ and
  $0$ otherwise.
  Note that this is actually a ``block permutation'' matrix.
\end{enumerate}

\subsection{Example: $S_4\subset S_5$}

Let's look at $S_4$ as a subset of $S_5$, and see what the natural
representation induces. Decompose $S_5$ as

I originally thought of a $5$-cycle in class, but Mark's better sense prevailed
and we used $(15)$.

$$S_5 = S_4 \cup (15)S_4\cup (25) S_4\cup (35)S_4 \cup (45)S_4.$$

\end{document}
\documentclass{article}
\usepackage[utf8]{inputenc}
\usepackage{amsthm,amsmath,amsfonts,amssymb}

\usepackage[margin=0.5in]{geometry}

\newcommand{\I}[1]{\mathcal{I}(#1)}
\newcommand{\V}[1]{\mathcal{V}(#1)}
\newcommand{\kn}{k[x_1,x_2,\ldots,x_n]}
\newcommand{\Rn}{R[x_1,x_2,\ldots,x_n]}
\newcommand{\Af}{{\Bbb A}^n}
\newcommand{\Ak}{{\Bbb A}_k^n}
\newcommand{\Akn}[1]{{\Bbb A}_k^#1}
\newcommand{\ARn}[1]{{\Bbb A}_{\Bbb R}^#1}
\newcommand{\An}[1]{{\Bbb A}^#1}
\newcommand{\fr}[1]{\mathfrak #1}
\newcommand{\C}{\mathbb C}
\newcommand{\Z}{\mathbb Z}
\newcommand{\Zn}[1]{\mathbb Z_{#1}}
\newcommand{\Q}{\mathbb Q}
\newcommand{\R}{\mathbb R}
\newcommand{\HH}{\mathbb H}
\newcommand{\F}[1]{\mathbb F_#1}
\newcommand{\chn}[1]{C_\bullet(#1)}
\newcommand{\bkt}[1]{\langle #1 \rangle}
\newcommand{\ttm}[4]{\left[
  \begin{array}{ c c }
     #1 & #2 \\
     #3 & #4
  \end{array} \right]}
\newcommand{\ab}{\text{ab}}
\newcommand{\ut}[1]{{#1}^\times}

\DeclareMathOperator{\im}{im}
\DeclareMathOperator{\Aut}{Aut}
\DeclareMathOperator{\End}{End}
\DeclareMathOperator{\Br}{Br}

\newcommand{\fix}[1]{\text{Fix}(#1)}

\theoremstyle{definition}
\newtheorem*{lem}{Lemma}
\newtheorem*{prop}{Proposition}

\newtheorem*{thm}{Theorem}
\newtheorem*{cor}{Corollary}
\newtheorem*{example}{Example}
\newtheorem*{defn}{Definition}

\title{Brauer groups}
\author{Don, transcribed by Soham}
\date{July 2016}

\begin{document}
\maketitle
 
\begin{defn}
 An \textit{algebra} is a vector space with an associative unital multiplication
 distributing over $+$.
\end{defn}

Equivalently, (why?)

\begin{defn}
 An algebra is a ring with a specifically chosen subfield in its center. 
\end{defn}

\section{Tensor products}
\begin{defn}
 A tensor product of two algebras $A$ and $B$ is an algebra $A\otimes B$ with
 underlying vector space $A\otimes B$, and multiplication
$$(a_1\otimes b_1)(a_2\otimes b_2) := (a_1a_2)\otimes(b_1b_2).$$
\end{defn}

\begin{lem}
 Let $M_i(K)$ be the $K$-algebra of $i\times i$ matrices with entries in $K$.
 Then, $\forall m, n>0$,

 $$M_n(K)\otimes_K M_m(K) \cong M_{nm}(K)$$
\end{lem}
\begin{proof}
  Note that $M_i(K) \cong \End_K(K^i)$. Given
  $$\phi\in M_m(K),\psi\in M_n(K),$$
  $(\phi,\psi)$ gives an element of
  $$\End_K(K^m\otimes_K K^n) = \End_K(K^{mn}).$$ We get a map
  $$h:M_m(K)\otimes M_n(K) \to M_{mn}(K)$$
  which is injective by construction (check) and surjective by dimension reasons.
\end{proof}

\subsection{Extension of scalars, a.k.a base change}
Let $A$ be a $k$-algebra, and $F/k$ an extension of fields. Then $A\otimes_k F$
is an $F$-algebra (in the obvious way). 

\begin{example}
 $M_n(k)\otimes_k F \cong M_n(F).$ 
\end{example}

\subsection{Generalized quaternion algebras}
\begin{defn}
 Let $k$ be a field $(\text{char } k \neq 2)$. The \textit{generalized quaternion
   algebra} $(a,b)_k$ is the $4$-dimensional $k$-algebra with the basis
 $\{1,i,j,ij\}$, under the relations
 $$i^2 =a, j^2=b,ij=-ji.$$
\end{defn}

\begin{example}
  \begin{enumerate}
  \item $(1,1)_\R \cong M_2(\R)$. 
  \item $(-1,-1)_\R \cong \HH$.
  \end{enumerate}
\end{example}

\begin{thm}
 $\HH$ is not a matrix algebra. 
\end{thm}
\begin{proof}
 $\dim\HH>1$ and $\HH$ is a division ring. 
\end{proof}

\begin{prop}
  If $k$ is a field ($\text{char }k\neq 2$), either $(a,b)_k\cong M_2(k)$ or

  $$(a,b)_k\otimes_k k(\sqrt a) \cong M_2(k(\sqrt a)).$$
\end{prop}
\begin{proof}
  First, note that for $a,b,u\in\ut k$,

  $$(a,b)_k\cong(u^2a, b)_k \text{ and } (a,b)_k \cong (b,a)_k$$

  First assume, wlog, that $b$ is a square. Then $(a,b)_k \cong (1,a)_k \cong
  M_2(k)$ via the map
  $$1\mapsto\ttm 1 0 0 1, i\mapsto\ttm 1 0 0 {-1}, j\mapsto\ttm 0 a 1 0,
  \ij\mapsto \ttm 0 b {-1} 0.$$

  If $a,b$ are not square, the map $(a,b)_k\otimes_k k(\sqrt a) \to M_2(k(\sqrt
  a))$ is

  $$1\otimes 1\mapsto\ttm 1 0 0 1,
  i\otimes 1\mapsto\ttm 0 {\sqrt a} {\sqrt a} 0, 
  j\otimes 1\mapsto\ttm 0 b 1 0, 
  ij\otimes 1\mapsto\ttm {\sqrt a} 0 0 {b\sqrt a}$$

  Since $a$ and $b$ are not squares, the image is linearly independent, so, by
  dimension-counting, this map is an iso.
\end{proof}

\begin{defn}
 Let $A,B$ be $k$-algebras. $A$ is a \textit{twisted form} of $B$ if there is a finite
 extension $F/k$ with
 $$A\otimes_k F \cong B\otimes_k F.$$
\end{defn}

\begin{defn}
 An algebra $A$ over a field $K$ is \textit{Brauer} if $A$ is a twisted form of
 $M_n(k)$ for some $n$. 
\end{defn}

\begin{lem}
 Brauer algebras are finite-dimensional. 
\end{lem}
\begin{proof}
 $A\otimes F\cong M_n(F)$.
\end{proof}

\begin{defn}
  An algebra is \textit{simple} if it has no nontrivial ($2$-sided) ideals.
\end{defn}
\begin{defn}
 An algerba is \textit{central} if its center is just the base field. 
\end{defn}

\begin{thm}[Wedderburn-Artin]
  Let $A$ be an f.d. central simple $k$-algebra. Then there exists an f.d.
  division ring $D\supset k$ and $n>0$ such that $A\cong M_n(D)$, where $n$ is
  unique and $D$ is unique upto isomorphism.  
\end{thm}

\begin{thm}
 Finite-dimensional division rings over a field are Brauer algebras. 
\end{thm}


\begin{lem}
 If $k$ is algerbaically closed, all division rings over $k$ are $k$.
\end{lem}

\begin{defn}
 The \textit{Brauer group} of a field is the group of Brauer algebras over that
 field, under the operation $\otimes_k$,
 up to the equivalence $A\sim M_n(A)$ for all algebras $A$. Equivalently,
 $A\sim B$ if, for some $m,n>0$,
 $$A\otimes_k M_n(k)\cong B\otimes_kM_m(k).$$
\end{defn}

\begin{thm}
  If $k$ is algebraically closed, $\Br(k) = 0$.
\end{thm} 

\begin{thm}
 $\Br(\R)\cong\Z/2$, generated by $[(-1,-1)_\R]$. 
\end{thm}

\begin{proof}
  \begin{lem}
   Any Brauer group equivalence class has exactly one division algebra. 
  \end{lem}
  \begin{prop}
    $\R, \C,\HH $ are all the division rings over $\R$.
  \end{prop}
\end{proof}

\end{document}
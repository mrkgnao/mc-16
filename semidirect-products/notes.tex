\documentclass{article}

\input{/home/mrkgnao/math/preamble.tex}

\title{Semidirect products}
\begin{document}
\maketitle
 
If $G$ is built out of $H$ and $K$, we might want to require
\begin{enumerate}
\item $H\leq G$, $K\leq G$
\item $HK = \bkt{hk : h\in H, k\in K} = G$
\item $H\cup K = \{e\}$
\end{enumerate}

Let's try just ``taking the union'' of two group presentations.
$$\Z/2 \ast \Z/2 = \bkt{x,e|x^2=e} \ast \bkt{y,e|y^2=e} = \bkt{x,y|x^2=y^2=e}$$
which is, upsettingly, an infinite group (note that $(xy)^m \neq (xy)^n$
for $m\neq n$).

We'll let this lead us to a definition:
\begin{defn}
  If $H=\bkt{H_g | H_r}$ and $K=\bkt{K_g|K_r}$, then their \textit{free product} is 
  $$H\ast K:= \bkt{H_g\cup K_g | H_r\cup K_r, e_H=e_R}$$.
\end{defn}

Now note that the group we were hoping to get ($\Z/2\times\Z/2\cong V_4$, the
Klein four-group) is recoverable as a quotient:

$$\Z/2\ast\Z/2/\bkt{xy=yx} \cong V_4$$

(this is in fact abelianization, in this special case).

\begin{thm}
 If $G$ satisfies the three relations at the beginning, it is isomorphic to a
 quotient of $H\ast K$. 
\end{thm}

\begin{proof}
 By the first isomorphism theorem, this is equivalent to a surjective
 homomorphism going from $H\ast K$ to $G$. Define $\varphi: H\ast K\to G$ by
 defining it to be the identity
 $$\varphi(h) = h, \varphi(k) = k \; \forall h\in H, k\in K$$
 on the subgroups $H,K\leq G$. Now extend this to all other words in the group
 by imposing the homomorphism condition.

 This is easily checked to be surjective, and we are done.
\end{proof}

\subsection{The direct product}
Given $\{H_i\}_{1,\ldots,r}$, the direct product $\prod H_i$ is the group whose elements
are of the form $(h_i)_i, h_i\in H_i$, and multiplication is defined elementwise.

\begin{thm}
  Let $G=\prod G_i$. Then
  \begin{enumerate}
  \item The $G_i$  embed ``sanely'' into $G$:
    $$G_i\trianglelefteq G.$$
  \item Cardinalities work out.
    $$|G| = \prod |G_i|.$$
  \item The $G_i$ are recoverable from $G$:
    $$G_i \cong \frac{G}{G_1\times\cdots\times \hat{G_i}\times\cdots\times G_r}.$$
  \end{enumerate}
\end{thm}
\begin{proof}
 Well-known. 
\end{proof}

\begin{thm}
  If $H,K\trianglelefteq G$, $H\cap K = \{e\}$, and $HK = G$, then $G\cong
  H\times K$. 
\end{thm}
\begin{proof}
  Notice that
  \begin{align*}
    [h,k] &= hkh^{-1}k^{-1} \\
          &= h(kh^{-1}k^{-1}), \text{so } [h,k]\in H \\
          &= (hkh^{-1})k^{-1}, \text{so } [h,k]\in K \\
          &= e.
  \end{align*} 
  Now take $\varphi:H\times K\to G$ sending $(h,k)\mapsto hk$ which is a
  homomorphism because $H$ and $K$ commute, and an isomorphism because $HK = G$.
\end{proof}

\begin{thm}[Structure theorem for finitely generated abelian groups]
  If $G$ is an f.g. abelian group,
  $$G\simeq \Z^r \times \Z/p_1^{n_1} \times\cdots\times \Z/p_k^{n_k}$$
  where $r$ is unique and the $(p_i,n_i)$ are unique up to permutation.
\end{thm}

\section{Day 2}
\subsection{Random theorem about dihedral groups}

\begin{thm}
  Let $D_{4n}$ be the symmetries of an $n$-gon, $n$ odd. Then
$$D_{4n} \cong D_{2n} \times \Z/2$$
\end{thm}

\begin{proof}
  As is well-known,
  $$D_{4n} = \bkt{r,s|r^{2n} = s^2 = (sr)^2 = 1}$$
  Now consider the subgroups
  $$H=\bkt{r^2,s}, K=\bkt{r^n}\subset D_{4n}$$

  \begin{itemize}
  \item $H$ has index $2$, and is therefore normal. (The two cosets are $H$ and
    $rH$.)
  \item For $K$, note that $sr^ns = srssrssrss\cdots rsss = (srs)^ns$
  \end{itemize}
\end{proof}

Now suppose that $H\trianglelefteq G, K\leq G, H\cap K=\{e\}, HK\leq G$.

\begin{prop}
 Given these assumptions, every element of $HK$ can be written uniquely as $hk$
 for $h\in H, k\in k$. 
\end{prop}

\begin{proof}
 Note that $h_1k_1=h_2k_2\implies h_2^{-1}h_1 = k_2k_1^{-1} = e$, so we have
 uniqueness. For existence, consider

$$kh = khk^{-1}k =  h'k, h'\in H\trianglelefteq G$$

Given any long string, this gives us a way to ``pull all elements of $K$ to the
right''.

This gives $HK$ the same underlying set as $H\times K$, but the operation is
different, in order to sidestep the non-normality (wlog) of $K$:
$$(h_1k_1)\cdot(h_2k_2) := h_1(k_1h_2k_1^{-1})k_1k_2.$$
\end{proof}

\begin{defn}
The \textit{automorphism group} of a group $G$, $\Aut(G)$, is the group of
isomorphisms $G\to G$, where the operation is composition.  
\end{defn}
 
\begin{example}
 $\Aut(\Z/n\Z) = \ut {(\Z/n\Z)}$. 
\end{example}

\begin{thm}
 Let $H,K$ be groups, $\varphi:K\to\Aut(H)$ a group hom, $G$ be the set of pairs
 $(h,k)$. Define the operation
 $$(h,k)(h',k') = (h\varphi(k)(h'), kk').$$
 \begin{enumerate}
 \item This makes $G$ a group with order $|H|\cdot|K|$.
 \item $(H,e)$ and $(e,K)$ are subgroups (where one hopes the notation
   will be self-explanatory) that are isomorphic to $H$ and $K$ respectively.
 \item With this identification,
   \begin{enumerate}
   \item $H\trianglelefteq G$
   \item $H\cap K = \{e\}$
   \item For all $h\in H, k\in K$, $khk^{-1}=\varphi(k)(h)$
   \end{enumerate}
 \end{enumerate}
 We denote this group $H\rtimes_\varphi K$.
\end{thm}

\begin{proof}
  Most of the cases are ones where there is essentially only one way to proceed,
  and it is a good idea to run through them mentally. 
\end{proof}

\begin{thm}
  If $H\trianglelefteq G, K<G, H\cap K=1$, and if $\varphi:K\to\Aut(G)$ is
  conjugation, then $HK\cong H\rtimes K$.
\end{thm}

\begin{prop}
  If $H,K$ are groups, and $\varphi:K\to\Aut(G)$, TFAE:
  \begin{enumerate}
  \item The identity set-function $H\rtimes K\to H\times K$ is a homomorphism.
   
  \item $\varphi$ is the trivial map.

  \item $K\trianglelefteq H\rtimes K$.
  \end{enumerate}
\end{prop}

Let $H$ be abelian, $K=\Z/2$, let $\varphi:K\to\Aut(H)$ send $\varphi(1)$ to
$\varphi(1)(x) = x^{-1}$. This gives a way to make $H\rtimes \Z/2$, when $H$ is
cyclic, this is dihedral.

Also:
\begin{itemize}
\item $\Z/n\rtimes \Z/2\cong D_{n}$
\end{itemize}

If $|G|=pq$, $p<q$ both prime, then

$$\exists P\leq G, Q\trianglelefteq G, |P|=p,|Q|=q$$
(by Sylow theorems), $P\cap Q=\{e\}$, so

$$G\cong Q\rtimes P\cong \Z/q\rtimes\Z/p$$

We need a $\phi:\Z/p\to \Aut(\Z/q)$, which exists iff $p|q-1$. If not, $G\cong \Z/pq$.

\subsection{Amalgamated free products}
Let $H\leq G, K\leq G, H\cap K = N'$. 

$$
\begin{tikzcd}
  N' \arrow[r] \arrow[d] & K\arrow[d,  "f_2"] \\
  H \arrow[r, "f_1"] &H\ast K/N
\end{tikzcd} 
$$

where
$$N=\bkt{f_1(n)f_2(n)^{-1} | n\in N'}$$

Now let $H\trianglelefteq G, K\twoheadleftarrow G, H\cap K=\{e\}$. ???

Consider short exact sequences of the form
$$1\longrightarrow A\overset{i}{\longrightarrow} E\overset \pi\longrightarrow G
\longrightarrow 1.$$

with $G,A,E$ finite, $A$ abelian. This is called an extension of $G$ by $A$. \\
Given this, we get an action of $G$ on $A$ as follows:

For each $g\in G$, let $e_g=\pi^{-1}(g)$. $e_g$ acts on $i(A)$ by conjugation
$e_gi(a)e_g^{-1}$ since $i(A)\trianglelefteq E$ and $A\simeq i(A)$, so $e_g$
acts on $A$.

Any other element in $\pi^{-1}(g)$ is of the form $e_gi(a)$ since

$$e_gi(a')i(a)(e_gi(a'))^{-1} = e_gi(a'a{a'}^{-1})e_g^{-1} = e_gi(a)e_g^{-1}$$

Given two extensions, they are equivalent if $\exists \beta:E_1\to E_2$ such
that

\begin{tikzcd}
  1\arrow[r] &A\arrow[r]{i_1}\arrow[d,equal] &E_1\arrow[r]{\pi_1}\arrow[d]{\beta} &G\arrow[r]\arrow[d,equal] &1\\
  1\arrow[r] &A\arrow[r]{i_2} &E_2\arrow[r]{\pi_2} &G\arrow[r] &1
\end{tikzcd} 

commutes.

\begin{prop}
 Equivalent extensions give the same $G$-module structure on $A$. 
\end{prop}

\begin{defn}
 Given a $G$-module $A$, a function $f:G\times G\to A$ is a \textit{$2$-cocycle}
 if
 $$f(g,h) - g\cdot f(h,k) + f(gh,k) - f(g,hk) = 0.$$ 
\end{defn}

\begin{defn}
 Given a $G$-module $A$, a function $f:G\times G\to A$ is a \textit{$2$-coboundary}
 if $\exists f_1:G\to A$ such that
 $$f(g,h) = g\cdot f_1(h) - f_1(gh) + f_1(g).$$ 
\end{defn}

Note that $B^2(G,A)\trianglelefteq H^2(G,A)$. We define, as one would expect,
the second group cohomology

$$H^2(G,A) \cong \frac{Z^2(G,A)}{B^2(G,A)}.$$


\begin{thm}
  There exists a bijection
  $$\{\text{equivalence classes of extensions } E\}\longleftrightarrow H^2(G,A)$$
  and the identity of $H^2(G,A)$ corresponds to direct products.
\end{thm}
\end{document}
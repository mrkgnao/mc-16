\documentclass{article}
\usepackage[utf8]{inputenc}
\usepackage{amsthm,amsmath,amsfonts,amssymb}

\usepackage[margin=0.5in]{geometry}

\newcommand{\I}[1]{\mathcal{I}(#1)}
\newcommand{\V}[1]{\mathcal{V}(#1)}
\newcommand{\kn}{k[x_1,x_2,\ldots,x_n]}
\newcommand{\Rn}{R[x_1,x_2,\ldots,x_n]}
\newcommand{\Af}{{\Bbb A}^n}
\newcommand{\Ak}{{\Bbb A}_k^n}
\newcommand{\Akn}[1]{{\Bbb A}_k^#1}
\newcommand{\ARn}[1]{{\Bbb A}_{\Bbb R}^#1}
\newcommand{\An}[1]{{\Bbb A}^#1}
\newcommand{\fr}[1]{\mathfrak #1}
\newcommand{\C}{\mathbb C}
\newcommand{\Z}{\mathbb Z}
\newcommand{\Zn}[1]{\mathbb Z_{#1}}
\newcommand{\Q}{\mathbb Q}
\newcommand{\R}{\mathbb R}
\newcommand{\sh}[1]{#1^+}
\newcommand{\chn}[1]{C_\bullet(#1)}
\newcommand{\bkt}[1]{\langle #1 \rangle}
\newcommand{\F}{\mathcal F}

\newcommand{\ab}{\text{ab}}

\DeclareMathOperator{\im}{im}
\DeclareMathOperator{\Aut}{Aut}

\newcommand{\fix}[1]{\text{Fix}(#1)}

\theoremstyle{definition}
\newtheorem*{lem}{Lemma}
\newtheorem*{thm}{Theorem}
\newtheorem*{cor}{Corollary}
\newtheorem*{defn}{Definition}

\title{Sheafification}
\author{Soham}
\date{July 2016}

\begin{document}
\maketitle
 
Consider a presheaf $\F$ on a space $X$. We want to canonically associate a
sheaf $\sh\F$ to $X$. To do this, associate to every open set $U\subset X$
the set of ``formal sections'' of the form

\begin{align}
  f &:= (f_p \in \F_p)_{p\in U} \\
    &\text{ with } f_q = s_q \text{ at all } q\in V, \text{ for } p\in \text{open }U\subset V
\end{align}

The idea is to call a set of germs at points in $U$ a ``section''
if it locally corresponds to functions in the presheaf. Now call the rule that associates
$$ U \mapsto \{f_1,f_2,\ldots,\} $$
$\sh\F$.

\begin{thm}
$\sh\F$ is a sheaf.
\end{thm}

Let's verify all the properties. This will necessarily be a dull task.
\begin{thm}
  $\sh\F$ is a presheaf.
\end{thm} 
\begin{proof}
  \begin{enumerate}
  \item\textit{The trivial restriction does nothing to sections.} This is
    obvious, since the open subsets and the sections of $\F$ over the open
    subsets are the same for $U$ and . . . $U$. 
  \end{enumerate}
\end{proof}

\begin{thm}[The other part]
  
\end{thm}

\end{document}
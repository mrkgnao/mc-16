\documentclass{article}

\input{/home/mrkgnao/math/preamble.tex}

\title{Graph polynomials}
\author{Jeff/Mia and Mia/Sachi}
\date{August 2016}
\begin{document}
\maketitle

\section{The chromatic polynomial}
\begin{defn}
 A \textit{proper graph coloring} is an assignment of colors to vertices such
 that adjacent vertices are different colors. 
\end{defn}

\begin{question}
 Given $x$ colors, how many ways are there to color $G$? 
\end{question}
\begin{notation}
 $P_G(x)$ is defined to be the number of colorings of $G$ using up to $x$ colors. 
\end{notation}

For $K_n$, this is how it goes:
\begin{table}[h]
\centering
\label{my-label}
\begin{tabular}{l|lllllllll}
 colors &1  &2  &3  &$\cdots$ &$n-1$ &$n$ &$n+1$ &$\cdots$ &$x$  \\
  \hline
 \# colorings &0  &0  &0  &$\cdots$ &$0$ &$n!$ &$\binom {n+1} k n!$ &$\cdots$ &$\binom x n {n!}$ 
\end{tabular}
\end{table}

For a tree $T$, we have:
\begin{table}[h]
  \centering
  \label{tree-table}
  \begin{tabular}{l|lllll}
    colors &$1$ &$2$ &$3$ &$\cdots$ &$x$\\
    \hline
    colorings &$0$ &$1$ &$3\cdot 2^{n-1}$ &$\cdots$ &$x{(x-1)}^{n-1}$
  \end{tabular} 
\end{table}

Let $f(k)$ denote the number of different $k$-partitions of $G$. Then
$$P_G(x) = \sum_{k=1}^{|V|} f(k)\cdot x(x-1)\cdots(x-n+1)$$

``It's a polynomial in $x$!''

\section{Actually calculating that}
How can we simplify a graph?
\begin{itemize}
\item Delete an edge to get $G-e$.
\item Contract an edge to get $G/e$.
\end{itemize}

\begin{prop}
  The chromatic polynomial $P_G$ satisfies
  $$P_G = P_{G-e} - P_{G/e}.$$
\end{prop}

\section{The reliability polynomial}
What is the probability that $G$ remains connected after removing edges with
probability $1-p$?

Call this $R_G(p)$. We have that

\begin{thm}
$$R_G(p) = pR_{G/e}(p) + (1-p)R_{G-e}(p)$$
\end{thm}

In $G/e$, $e$ does not fail. In $G-e$, $e$ necessarily fails. This is
essentially what the proof consists of.

\end{document}
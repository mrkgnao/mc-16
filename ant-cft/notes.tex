\documentclass{article}

\usepackage[utf8]{inputenc}
\usepackage{amsthm,amsmath,amsfonts,amssymb}
\usepackage{enumerate}

\usepackage[margin=0.5in]{geometry}
\newcommand{\Mod}[1]{\ (\text{mod}\ #1)}

\newcommand\dif{\mathop{}\!\mathrm{d}}
\newcommand{\I}[1]{\mathcal{I}(#1)}
\newcommand{\V}[1]{\mathcal{V}(#1)}
\newcommand{\kn}{k[x_1,x_2,\ldots,x_n]}
\newcommand{\Rn}{R[x_1,x_2,\ldots,x_n]}
\newcommand{\Af}{{\Bbb A}^n}
\newcommand{\Ak}{{\Bbb A}_k^n}
\newcommand{\Akn}[1]{{\Bbb A}_k^#1}
\newcommand{\ARn}[1]{{\Bbb A}_{\Bbb R}^#1}
\newcommand{\An}[1]{{\Bbb A}^#1}
\newcommand{\fr}[1]{\mathfrak #1}
\newcommand{\C}{\mathbb C}
\newcommand{\Z}{\mathbb Z}
\newcommand{\Zn}[1]{\mathbb Z_{#1}}
\newcommand{\Q}{\mathbb Q}
\newcommand{\R}{\mathbb R}
\newcommand{\F}[1]{\mathbb F_#1}
\newcommand{\Fpn}[2]{\mathbb F_{#1^#2}}
\newcommand{\chn}[1]{C_\bullet(#1)}
\newcommand{\bkt}[1]{\langle #1 \rangle}
\newcommand{\gal}[2]{\text{Gal}(#1/#2)}
\newcommand{\iv}[1]{#1^{-1}}
\newcommand{\oo}[1]{\mathcal O_{#1}}
\newcommand{\pp}{\mathfrak p}
\newcommand{\mm}{\mathfrak m}
\newcommand{\ok}{\mathcal O_K}
\newcommand{\ol}{\mathcal O_L}
\newcommand{\tr}[2][L/K]{\text{Tr}_{#1}(#2)}
\newcommand{\nm}[2][L/K]{\text{Nm}_{#1}(#2)}
\newcommand{\Nm}[1]{\text{Nm}(#1)}
\newcommand{\Tr}{\text{Tr}}
\newcommand{\Id}[1]{\text{Id}(#1)}
\newcommand{\Cl}[1]{\text{Cl}(#1)}
\newcommand{\Prin}[1]{\text{Prin}(#1)}
\newcommand{\idok}{\Id \ok}
\newcommand{\clok}{\Cl \ok}
\newcommand{\clk}{\Cl K}
\newcommand{\ab}[1]{#1^{\text{ab}}}
\newcommand{\ut}[1]{#1^\times}
\newcommand{\ur}[1]{#1^{\text{ur}}}
\newcommand{\prinok}{\Prin \ok}

\DeclareMathOperator{\im}{im}
\DeclareMathOperator{\Aut}{Aut}
\DeclareMathOperator{\Frac}{Frac}

\newcommand{\galqp}{\gal {\overline{\Q_p}} {\Q_p}}
\newcommand{\galfp}{\gal {\overline{\F p}} {\F p}}
\newcommand{\fix}[1]{\text{Fix}(#1)}
\newcommand{\kp}{K_\pp}

\theoremstyle{definition}
\newtheorem*{lem}{Lemma}
\newtheorem*{thm}{Theorem}
\newtheorem*{cor}{Corollary}
\newtheorem*{defn}{Definition}
\newtheorem*{example}{Example}

\title{Algebraic number theory}
\author{Soham}
\date{July 2016}

\begin{document}
\maketitle
 
\section{Day 1}
\begin{defn}
 A \textit{number field} is a finite extension of $\Q$.
\end{defn}

Some examples of number fields are:
\begin{enumerate}
\item $\Q$
\item $\Q(i)$
\item $\Q(\sqrt 2)$
\item $\Q(\sqrt 2, \sqrt 3)$
\end{enumerate}

Note that the last example is not a primitive extension.
\begin{thm}[Primitive element theorem] 
  If $K/\Q$ is a finite extension, then $\exists \alpha\in K$ with $K=\Q(\alpha)$.
\end{thm}

\subsection{Invariants of a number field}
\begin{enumerate}
\item $\gal K\Q$
\item $[K:\Q]$
\item ???
\end{enumerate}

Note that our extensions will not always be Galois. We get separability for free
since we are working over $\Q$, which has characteristic zero, but we might not
have normality.

\subsection{The ring of integers}
\begin{defn}
 If $\alpha\in K$, we say that $\alpha$ is an \textit{algebraic integer} if it
 satisfies a monic polynomial with integer coefficients. 
\end{defn}
\begin{defn}
 The (a priori) set of algebraic integers in $K$ is denoted $\ok$.
\end{defn}

\begin{defn}
 The \textit{discriminant} of a number field is an invariant associated to $\ok$
 that measures how ``complicated'' or ``big'' $K$ is. 
\end{defn}

\subsection{Traces and norms}
Let $L/K$ be a field extension, and $\beta\in L$. We have a natural map

\begin{align*}
  m_\beta : &L\to L \\
            &x\mapsto \beta x
\end{align*}

which is linear, and hence we have maps

$$\text{Tr}_{L/K}: L\to K, \text{Nm}_{L/K}: L\to K$$

corresponding to the trace and the determinant respectively.

Fact: $\nm {\alpha\beta} = \nm {\alpha} \nm{\beta}$

Choose a basis $\beta_i$ for $\ok$ (as a $\Z$-module.)
The discriminant $\Delta_K = \det({\tr [K/\Q] {\beta_i\beta_j}})$ is another invariant.

\begin{thm}
 Given $n, M$ positive integers, there are finitely many number fields $K$ with
 $[K:\Q] = n$ with $\Delta_K = M$. 
\end{thm}

\begin{thm}
  If $B\subseteq\ok$ is a subring, define $\Delta_B$ similarly.
  $\Delta_B\neq 0 \implies B$ has finite index in $\ok$. Also,
  $$\Delta_B = [\ok:B]^2\cdot\Delta_K$$
  so we can see whether we've ``found'' the whole ring of integers by looking at
  whether $\Delta_B$ is squarefree, and, if not, we know what index the subgroup
  we're looking for has. 
\end{thm}

\begin{thm}[Dirichlet's unit theorem]
  If $K$ is a number field of signature $(r,s)$, we have that
  $$\ok^{\times} \cong \mu(K) \times \Z^{r+s-1}$$
  where the signature is defined by letting $r$ be 
\end{thm}

\section{Day 2}
\subsection{The trace pairing}
The map

\begin{align*}
  \text{Tr}: &L\times L\to K \\
             &(x,y) \mapsto \tr {xy}
\end{align*}

is bilinear.

Recall that there is a natural map $V\times V^{\ast} \to K$ for $V$ a vector space.

Now, with the trace map in hand, we get a map
\begin{align*}
  &V\to V^{\ast}\\
  &v\mapsto\text{Tr}(v,\textunderscore)
\end{align*}

\subsection{A legit non-noetherian ring}
$\overline \Z$ is not noetherian: consider
$$(\sqrt 2) \subset (\sqrt[4]2) \subset (\sqrt[8]2) \subset\cdots$$

\subsection{Brief digression}
\begin{defn}
  A \textit{Dedekind domain} is an integral domain $R$ such that
  \begin{enumerate}
  \item $R$ is noetherian
  \item $R$ is integrally closed (in its field of fractions) 
  \item every nonzero prime ideal is maximal 
  \end{enumerate}
\end{defn}

\begin{thm}
  A discrete valuation ring (henceforth DVR) is a PID with a unique maximal
  ideal.
\end{thm}
Facts:
\begin{enumerate}
\item DVRs are Dedekind domains.
\item Guess: You have a valuation from the power of the generator of the maximal ideal
  dividing an element.
\end{enumerate}


\begin{thm}
  If $P$ is a prime ideal and $R$ is a Dedekind domain, $R_P$ is a DVR.
\end{thm}

$\ok$ is a Dedekind domain.
Dedekind domains have unique factorization of ideals.

\section{Day 3}
\subsection{Norms of ideals}

We want a definition for $\Nm I$ for $I$ an ideal of $\ok$, satisfying the
following ``common-sense'' properties:

\begin{enumerate}
\item $\Nm {IJ} = \Nm I \Nm J$
\item $\Nm \ok = 1$
\item If $\beta\in K$ then $\Nm \beta = \Nm{(\beta)}$
\end{enumerate}

If $p\ok = P_1^{e_i}\cdots P_g^{e_g}$, we just need to define $\Nm {P_i}$. Note
that, with $p\in K$ and $m=[K:\Q]$,

$$\Nm p = p^m$$

(the matrix associated to $p$ is just $pI_m$).

We need $\Pi^g_{i=1} {\Nm P_i}^{e_i} = p^m$. Recall Swapnil's postulate

$$\sum_{i=1}^g e_if_i = m$$

which suggests the definition

$$\Nm {P_i} = p^{f_i}$$

or, alternatively,

$$\Nm I = [\ok:I]$$

\subsection{Fractional ideals}
\begin{defn}
 A \textit{fractional ideal} in $\ok$ is a nonzero, finitely generated
 $\ok$-submodule of $K$.
\end{defn}

Fact/claim: any fractional ideal has the form $\frac1d I$ for some $d\in\ok$ and
$I\subseteq\ok$ an ideal.

For example, $\Z \supset J:= \frac13(5)$ is fractional, with $J \cdot (3) = (5)$.

\begin{defn}[and theorem]
  The set $\idok$ of fractional ideals of $\ok$ is a group, freely generated by
  the prime ideals of $\ok$.
\end{defn}

\begin{defn}
 A \textit{principal fractional ideal}  is one of the form $\alpha\ok$ for
 $\alpha\in K^\times$.
\end{defn}

Let $\prinok$ denote the subgroup of principal fractional ideals. If $I$ is a
fractional ideal, define
$$\iv I = \{\alpha\in K: \alpha I\subseteq \ok\}$$

\begin{defn}
  The class group of $\ok$ is

  $$\clok := \idok/\prinok.$$
\end{defn}

\begin{thm}[Minkowski]
 There is a set of representatives for the ideal class group consisting of
 integral ideals $I$ with
 $$\Nm I \leq \frac{n!}{n^n}\left( \frac 4\pi \right)^s |\Delta_K|^{1/2}$$
\end{thm}

\begin{example}
  $K=\Q(\sqrt{-14})$. The Minkowski bound is

  $$\frac{2!}{2^2}\frac4\pi\sqrt{56} \approx 4.8 < 5$$
\end{example}

\section{Day 5}

\subsection{Hilbert class fields}
Let $K$ be a number field with class group $\clk$, satisfying the following:
\begin{itemize}
\item $\gal L K$ is abelian.
\item Every prime $\mathfrak p$ in $K$ is unramified in $L$.
\end{itemize}

For such an \textit{unramified abelian} extension, we have

$$\gal L K \cong \clk.$$

Moreover, if $H \subseteq \clk$, then the subfield of $L$ corresponding to $H$
under the Galois correspondence is the one where every ideal in $H$ splits completely.

\begin{example}$K=\Q(\sqrt{-5}), L=\Q(\sqrt{-5},\sqrt{-1})$.
  First note that $\Delta_K = -20$, and (with some more work) that $\Delta_L = -400$.
  We have that
  \begin{itemize}
  \item $(2) = (2,1+\sqrt{-5}) = {(\alpha^3+2\alpha+1)}^2$
  \item $(5) = (\sqrt{-5}) = {(-\alpha^3+\alpha^2-2\alpha+2)}^2{(\alpha^3+\alpha^2+2\alpha+2)}^2$
  \end{itemize}

  Notice that we made both ideals principal. This always happens:
  \begin{thm}
    If $I\subset\ok$ is an ideal, then $I\ol$ is principal.
  \end{thm}
\end{example}

Start with a number field $K$, and take its Hilbert class field $L_0$. Then pass to
the class field of \textit{that}, $L_1$. Does this tower ever stabilize?

No: $K = \Q(\sqrt{-2\cdot 3\cdot5\cdot7\cdot11\cdot13})$ is a counterexample. (Somehow.)

\subsection{Moduli}
Define

$$\mathfrak m = \Pi_{\pp} {\pp}^{m(\pp)}$$

where we think of real and complex embeddings as ``infinite primes'', and let

$$m(\pp) \geq 0$$
$$m(\text{real embedding}) = 0\text{or}1$$
$$m(\text{complex embedding}) = 0$$

with finitely many exponents nonzero. Now, given $\mm$, let $S(\mm) = \{\pp:
m(\pp) > 0\}$.

Let $I^S$ be the product of the ideals not in $S$.
$$K_{\mm,1} = \{\alpha\in K^\times : \alpha\equiv 1 \Mod{\pp^{m(\pp)}},
\text{finite } \pp\}$$

where $\alpha_\pp > 0$ for infinite $\pp\in\mm$ or something.

\begin{defn}
  The \textit{ray class group} is defined to be
  $$C_m = I^{S(\mm)}/K_{\mm,1}.$$
\end{defn}

We make another definition.

\begin{defn}
 The \textit{ray class field} is a maximal abelian extension $L_\mm/K$
 unramified except at places in $S(\mm)$.
\end{defn}

$K=\Q(\sqrt 6)$. We have an exact sequence

$$0\to U/U_+ \to K^{\times}/K_+ \to C_\mm \to \clk \to 0.$$
some stuff here that I don't understand about $K$ having two infinite places
and a fundamental unit and $\Q(\sqrt{-2},\sqrt{-3})$ being the ``narrow class
field'' or something

\subsection{Completions}
If $\pp$ is a prime ideal, define the completion $K_\pp$ to be the set of Cauchy
sequences in $K$ under the $\pp$-adic metric

$$|\alpha|_\pp = {\nm[K/\Q] \pp}^{-v_{\pp}(\alpha)}$$

\subsubsection{Directed systems, inverse limits}
Let $I$ be a poset and $A$ be a directed $I$-indexed directed set, with maps
$f_{ij}:A_j\to A_i$ (we had a vote on which way the arrow should go).

\begin{itemize}
\item $\cdots \to \Z/{p^4} \to \Z/p^3 \to \Z/p^2\to\Z/p$.
\item $A_n = \Z/n\Z$, $I = \mathbb N_{>0}$ with the partial order being induced
  by divisibility.
\item Let $K$ be an algebraically non-closed field that is not $\overline K$. If
  $M\subseteq L\subseteq \Q$, then $\gal L K$ includes into $\gal M K$.
  In fact, there is an exact sequence
  $$0\to\gal L M\to \gal L K\to \gal M K \to 0.$$
\end{itemize}

Define the projective limit to be
$$\{(a_i)_{i\in I} : f_{ij}(a_j) = a_i \text{ for }i\leq j\}.$$

We have
\begin{itemize}
\item $\varprojlim \Z/p^n = \Z_p$.
\item $\varprojlim \Z/n = \hat\Z$.
\item We know that $\gal {\Fpn p n} {\F p} \cong \Z/n$ (with the
  Frobenius as the generator). We get
  $$\varprojlim \gal {\overline{\F p}} {\F p} = \hat\Z.$$
\end{itemize}

Also note that $K_\pp = \Frac(\varprojlim \ok/\pp^n)$.

If $L/K$ is an extension of number fields, we have an inclusion of groups
$L^\times \supset (L^{\text{ab}})^\times \supset K^\times$. As subgroups of
$K^\times$,

$$\nm[L/K] {\ut L} = \nm[\ab L/K] {\ut{(\ab L)}}$$

so ``norms cannot see nonabelian behavior''.

\subsection{Local class field theory}

\begin{thm}[Some kind of ``main theorem''?]
Let $K$ be a $p$-adic field, that is, a finite extension of $\Q_p$. We have
$$\gal {\ab K} K = \ab{\gal {\overline K} K} \supset \ab{W_K}\cong \ut K.$$
\end{thm}

Now we switch to $K$ for a number field and $K_\pp$ its completion at $\pp$.
(Why is this okay? Because we have
\begin{thm}[Krasner's lemma]
  Every $p$-adic field arises in this way.
\end{thm}
so it's fine.)

There is an exact sequence

$$0\to I_{\Q_p} \to \gal {\overline{\Q_p}} {\Q_p} \to \gal {\overline{\F p}} {\F
  p} \to 0$$

corresponding to the tower of fields

$${\overline K}_\pp / \ab {K_\pp} / K_\pp.$$

Now, $W_{\Q_p}$ is the preimage of $\Z\subset\hat\Z$ in $\galqp$.

Abelian extensions of $K_\pp$ are in bijection with finite index subgroups of
$\ut{K_\pp}$, where the map is
$$L\mapsto \nm[L/\kp]{\ut L}$$
and we have $\gal L \kp \cong {\ut \kp}/{\nm[L/\kp]{\ut L}}$.

\subsubsection{What are the degree $2$ extensions of $\Q_p$?}
Using the norm map $\ut{\Q_p}\to \Z$, we have an exact sequence
$$0\to \ut{\Z_p} \to \ut{\Q_p} \to \Z\to 0.$$

There are at least three:
\begin{itemize}
\item The image of ${(\ut{\Z_p})}^2$ in $\ut{Q_p}$ gives us one.
\item The preimage of $2\Z$ gives us another.
\item $\ldots$
\end{itemize}

\subsection{Galois cohomology}
Let $G = \gal L K$, $M$ be a $G$-module, that is, a $\Z[G]$-module (so an
abelian group with a ``sensible'' $G$-action).

We define a sequence of functors (?) $H^r$, $r$ is a nonnegative integer.

$$H^0(G,M) = M^G = \{m\in M: gm = m \forall g\in G\}.$$
$$H^0(G,L) = K$$
$$H^0(G,\ut L) = \ut K$$

Now suppose that $M$ = ``automorphisms of an $L$-structure''.

Morally, $H^1(G,M)$ measures how many ``$K$-structures'' there are that become
isomorphic to $M$ after base change from $K$ to $L$. (Tensor with $L$?)

$$H^1(G,L) = 0$$
$$H^1(G,\ut L) = 0$$ 

(The second generalizes to $GL_n(L)$.) What does $H^2(G,M)$ measure?

$$H^2(G,L) = 0$$
$$H^2(G,\ut L) = \Z/[L:K]$$

\subsection{The Brauer group}
In general, we call

$$H^2(\gal {\overline K} K, \ut{\overline K})$$

the Brauer group of $K$, and there is a bijection between the Brauer group and
the set of central simple $K$-algebras up to equivalence.

$K$ $p$-adic,
$$H^2(\gal {\overline K} K, \ut{\overline K})\cong\Q/\Z$$

For $K$ a number field,

$$0\to H^2(\gal {\overline K} K, \ut{\overline K}) \to \bigoplus_{\pp \text{ prime}} H^2(\gal {\overline {K_\pp}} {K_\pp},
\ut{\overline {K_\pp}}) \to \Q/\Z \to 0$$

A \textit{quaternion algebra} over a number field $K$ looks like $\mathbb H_{a,b}$.

\subsection{Cohomology!}
Define

$$C^r(G,M) = \{\text{maps } G^r\to M\}$$

and boundary maps

$$d:C^r(G,M)\to C^{r+1}(G,M)$$

defined in such a way that $d^2 = 0$.

\begin{example}
 a 
\end{example}

We get a chain complex

$$\cdots\to C^{r-1}(G,M)\to C^r(G,M)\to C^{r+1}(G,M) \to \cdots$$

and the \textit{group homology} is defined to be

$$H^r(G,M) = \ker d/\im d$$

where we choose $d$ ``sensibly'', to quote Alfonso.

\subsubsection{Another perspective}

Let

$$0\to A\to B\to C\to 0$$

be an exact sequence of $G$-modules. Applying the ``fixed-points'' functor, we
get

$$0\to A^G\to B^G\to C^G$$

and this sequence continues to the left using the right derived functors of
$\_^G$, which are precisely group cohomology!

$$0\to A^G\to B^G\to C^G\to H^1(G,A)\to H^1(G,B)\to H^1(G,C)\to H^2(G,A)\to\cdots$$

We also have

$$H^2(G,M)\cong\text{Ext}^1(G,M)$$

so $H^2$ classifies all $E$s that fit into exact sequences $0\to M\to E\to G\to 0$.

There is a cup product map

$$H^r_T(G,M)\times H^s_T(G,N)\to H^{r+s}_T(G,M\otimes N)$$

such that


\section{The ``difficult'' integral}
Define

$$\zeta_K(s) = \sum_{I\neq 0} \frac1{{N(I)}^s} = \prod_{\text{prime } P}\frac1{1-{N(P)^{-s}}}$$
$$L(\chi,s) = \sum_{I\neq 0} \frac{\chi(s)}{{N(I)}^s} = \prod_{\text{prime } P}\frac1{1-{\chi(P)}{N(P)^{-s}}}$$
$$\chi_{\infty}(\alpha) = \begin{cases}
  1 &\text{if } N(\alpha) > 0 \\
  -1 &\text{if } N(\alpha) < 0
  \end{cases}$$

\section{Attempt $\#1$}
Let

$$F(a) = \int_0^1 \frac{\log(1+x^a)}{1+x} \dif x$$

Let's try parts:

\begin{align*}
  F(a) &= \log(1+x^a)\log(1+x)\big|^1_0 -
\int_0^1\frac{\log(1+x)}{1+x^a}ax^{a-1}\dif x\\
       &= (\log 2)^2 - F(1/a)
\end{align*} 

so

$$F(a) + F(1/a) = (\log 2)^2$$

Now we expand the integrand of $F$ as a series:
\begin{align*}
F(a) &= \int_0^1 \left( \sum_{k=1}^\infty (-1)^{k-1} x^{k-1} \right) \left(
  \sum_{n=1}^\infty \frac{(-1)^n x^{na}}n \right) \dif x \\
     &= \int_0^1 \sum_{n,k=1}^\infty (-1)^{n+k} \frac{x^{an+k-1}}n \dif x \\
     &= \sum_{n,k=1}^\infty (-1)^{n+k} \frac{x^{an+k}}{n(an+k)}\dif x \\
     &= \sum_{n,k=1}^\infty \frac{(-1)^{n+k}} {n(an+k)}\dif x     
\end{align*}

Our integral is $F(w) = F(2+\sqrt 3)$. Note that $\bar w = 1/w$, so that

$$F(w) - F(\bar w) &= \sum_{n,k=1}^\infty (-1)^{n+k} \frac{-2\sqrt 3}{n^2+4nk+k^2} $$

$$T=\frac12$$


\end{document}
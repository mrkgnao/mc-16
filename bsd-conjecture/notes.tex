\documentclass{article}

\input{/home/mrkgnao/math/preamble.tex}

\title{The BSD conjecture}
\author{David, transcribed by Soham}
\date{August 2016}
\begin{document}
\maketitle

\begin{thm}[Mordell-Weil]
 $E(\Q)$ is a finitely generated abelian group. 
\end{thm}

This gives us that
$$E(\Q) = T\oplus \Z^r$$
where $T$ is torsion, and we know about its structure due to a (very cool) theorem of Mazur.

\begin{thm}[Hasse]
  If $E/\F_p$ is an elliptic curve, define
  $$a_p = p+1-\#E(\F_p).$$
  Then $|a_p|\leq 2\sqrt p$.
\end{thm}

One expects half ....

Recall that

$$\Delta = -16(4a^3+27b^2)$$

is the discriminant.

If $E$ has bad reduction mod $p$, there are three cases:
\begin{enumerate}
\item A triple point, then the group of points (leaving out the singular point)
  is isomorphic to the additive group $\F_p$. ``Additive''. $a_p = 0$.
\item A double point, with tangent lines having slope in $\F_p$. ``Split
  multiplicative.'' $a_p = 1$.
\item A double point, with tangent lines having slope in $\F_{p^2}\\\F_p$.
  ``Nonsplit multiplicative''. $a_p = -1$.
\end{enumerate}

Algorithms that decide which of these is the case do exist.

\section{$L$-functions}
\begin{defn}
  The $L$-function of $E$ is
  $$L(E,s) := \prod_{p|\Delta} \frac1{1-a_pp^{-s}}
              \prod_{p\not|\Delta} \frac1{1-a_pp^{-s}+p\cdot p^{-2s}} =
              \sum_{n=1}^\infty \frac{a_n}{n^s}$$
\end{defn}

This ``degree-$2$ $L$-function'' converges for $\real(s)>\frac32$. Analytically
continue?

We want a ``completed'' $L$-function that has a nice functional equation. Recall
that

$$\Gamma(z) = \int_0^\infty t^{z-1}e^{-t} \dif t$$

Define the \textit{conductor} of $E$ to be
$$N = \prod_{p|\Delta}p^{e_p}$$
where $e_p = 1$ if multiplicative reduction, $2$ if additive reduction (these
are for $p\geq 5$).

$\varepsilon =\pm 1$ is the ``root number''.

Define

$$\Lambda(E,s) = N^{s/2}{2\pi}^{-s}\Gamma(s)L(E,s)$$

\begin{thm}
  $\Lambda(E,s)$ analytically continues to all of $\C$, and
  $$\Lambda(E,2-s) = \varepsilon\Lambda(E,s).$$
\end{thm}

\begin{proof}
 $f_E = \sum a_nq^n$, $q=\exp(2\pi i nz) \in S_0(\Gamma_0(N))$. 
 Now a result of Hecke shows that
 $L(f_E,s)=\sum a_nn^{-s}$, the $L$-function of the modular form, has an
 analytic continuation and satisfies a functional equation.
\end{proof}

\begin{conj}[BSD]
  If $L(E,s) = \alpha_m(s-1)^m + \cdots$ is the Taylor series around $s=1$, then
  $$\text{(analytic rank) }m= r \text{ (algebraic rank)}$$ 
  and
  $$\alpha = \frac{\Omega_E \Reg(E) \#\shah(E/\Q)\cdot \prod_p c_p}{(#T)^2}$$
  where
  $$\Omega_E = \int_{E(\R)}\frac{\dif x}{2y+a_1x+a_3}$$
  is a \textit{period} (somehow related to de Rham!), writing $y^2+a_1xy+a_3y =
  x^3+a_2x^2+a_4x+a_6$. (``Minimal model'', minimal discriminant.)
\end{conj}

\subsection{Heights}


\section{Moar cohomology, because why not}
We have a short exact sequence over $\bar\Q$:
$$0\to E[n]\to E \overset{n}{\to} E\to 0$$

and, taking fixed-points, we get a long exact sequence in cohomology

\end{document}
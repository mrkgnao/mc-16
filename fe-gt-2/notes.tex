\documentclass{article}
\usepackage[utf8]{inputenc}
\usepackage{amsthm,amsmath,amsfonts,amssymb}

\usepackage[margin=0.5in]{geometry}

\newcommand{\I}[1]{\mathcal{I}(#1)}
\newcommand{\V}[1]{\mathcal{V}(#1)}
\newcommand{\kn}{k[x_1,x_2,\ldots,x_n]}
\newcommand{\Rn}{R[x_1,x_2,\ldots,x_n]}
\newcommand{\Af}{{\Bbb A}^n}
\newcommand{\Ak}{{\Bbb A}_k^n}
\newcommand{\Akn}[1]{{\Bbb A}_k^#1}
\newcommand{\ARn}[1]{{\Bbb A}_{\Bbb R}^#1}
\newcommand{\An}[1]{{\Bbb A}^#1}
\newcommand{\fr}[1]{\mathfrak #1}
\newcommand{\C}{\mathbb C}
\newcommand{\Z}{\mathbb Z}
\newcommand{\Zn}[1]{\mathbb Z_{#1}}
\newcommand{\Q}{\mathbb Q}
\newcommand{\R}{\mathbb R}
\newcommand{\sh}[1]{#1^+}
\newcommand{\chn}[1]{C_\bullet(#1)}
\newcommand{\bkt}[1]{\langle #1 \rangle}
\newcommand{\gal}[2]{\text{Gal}(#1/#2)}
\newcommand{\iv}[1]{#1^{-1}}
\newcommand{\F}{\mathcal F}

\newcommand{\ab}{\text{ab}}

\DeclareMathOperator{\im}{im}
\DeclareMathOperator{\Aut}{Aut}

\newcommand{\fix}[1]{\text{Fix}(#1)}

\theoremstyle{definition}
\newtheorem*{lem}{Lemma}
\newtheorem*{thm}{Theorem}
\newtheorem*{cor}{Corollary}
\newtheorem*{defn}{Definition}
\newtheorem*{prop}{Proposition}

\title{Galois theory notes, week 2, day 1 of 4}
\author{Soham}
\date{July 2016}

\begin{document}
\maketitle
\begin{enumerate}
\item Recall that when we have a field automorphism $L/K$, we define the group of
$K$-automorphisms of $L$ and denote it $G_{L/K}$. When this extension is Galois
($=$ normal and separable), it is called the Galois group $\text{Gal}(L/K)$.

\item If $H$ is a finite group of automorphisms of $L$, write $L^H$ for the fixed
field of $H$.

\item If $[L:K]$ is finite, then $G_{L/K} \leq [L:K]$.
\end{enumerate}

\begin{thm}
If $L$ is any field and $H$ a finite group of auts of $L$, then $[L:F_H] = |H|$.
\end{thm}

\begin{proof}
 We'll show $[L:F_H] \leq |H| =: s$ by showing that any set of $s+1$ elements of
 $L$ are linearly dependent over $F_H$.

 Once we have this, the other direction follows from the last theorem from last
 week, and we get that

 $$|H| \leq [L:F_H]$$

 Suppose $a_1,\ldots,a_{s+1}\in L$. We want to see if they can be linearly
 independent over $F_H$.

 Consider all the images of the $a_i$ under the elements of $H$:

 $$v_i := (\sigma_1(a_i),\sigma_2(a_2),\ldots,\sigma_s(a_i))^t$$

 which is an $s+1$-set in $L^s$, so it must be linearly dependent over $L$.
 Rearranging, we get

 \begin{align}
  \sigma_i(a_{s+1}) = b_1\sigma_i(a_1) + \cdots + b_s\sigma_i(a_s) 
 \end{align}

 Apply, say, $\sigma := \sigma_k \in H$.

 \begin{align}
   (\sigma \sigma_i)(a_{s+1}) = \sigma(b_1)(\sigma \sigma_i)(a_1)+\cdots+\sigma(b_s)(\sigma\sigma_i)(b_s)
 \end{align}

 but these are just the old equations in some other order (multiplication by a
 group element is an automorphism, so is, in particular, bijective) and we get a
 smaller dependence relation.

So what we should have done is started with a minimal dependence relation among
some of the vectors $v_i$. By the same argument, we get
\begin{align}
  v_k &= \sum b_iv_i \\
  v_k &= \sum \sigma(b_i)v_i
\end{align}

By the minimality of the dependence relation, $b_i - \sigma(b_i)$ is zero, so
the $b_i$ are linearly dependent over $F_H$.

Look at the first coordinate of each vector. This tells us that
$a_1,a_2,\cdots,a_k$ are linearly dependent over $F_H$, so extending to the full
set of $s+1$ vectors does the same thing.
\end{proof}

\section{Splitting fields}

A \textit{splitting field}  of a nonconstant $f\in k[x]$ is an extension $L/k$
such that
\begin{enumerate}
\item $f$ splits in $L[x]$
\item $L = k[a_1,\ldots,a_d]$
  where the $a_i$ are the roots of $f$ in $L$
\end{enumerate}

\begin{thm}
 Splitting fields exist. 
\end{thm}
\begin{proof}
  Take an irred factor $g$ of $f$. Then
  $$k \subseteq k[x]/(g) =: k_1$$
  is a field extension in which $g$ has a root $a_1 := \bar x$.

  In $k_1[x]$, $f(x) = (x-a_1)f_1(x)$. Now $f_1$ is a lower-degree polynomial. Repeat.
\end{proof}

\begin{lem}[``Fundamental lemma'']
  If
  \begin{enumerate}
  \item $k$ and $K$ are fields which are isomorphic under $\varphi:k\to K$
  \item $f\in k[x]$, $F\in K[x]$
  \item $l = spl_k(f), L = spl_K(F)$
  \end{enumerate}
  then there is an isomorphism $l\to L$ extending $\varphi$.
\end{lem}


\section{Day 2}
\begin{thm}[Different definitions of normality]
  For a field extension $L/k$ of finite degree, TFAE:
  \begin{enumerate}
  \item $L$ is a splitting field of some $f\in k[x]$
  \item If $g\in k[x]$ is irred and has a root in $L$, then $g$ splits in $L$. 
  \end{enumerate}
\end{thm}
\begin{proof}
  \begin{enumerate}
  \item $[L:k]$ is finite, so we can pick a finite basis $\{a_i\}_i$ of $L$ over
    $k$. Let $g_i = min_k(a_i)$ be the minimal polynomial of $a_i$ over $k$,
    which split in $L$.

    Now set $g = \Pi g_i$. $f$ splits in $L[x]$ and, among possibly others, its
    roots include the $a_i$. So $L$ is a splitting field for $f$. (What about
    minimality of the splitting field?)
    
  \item Now we assume that $L$ is a splitting field of $f\in k[x]$. Say $g\in
    k[x]$ is irred and has a root $a\in L$.

    Take a splitting field $S$ (ugh) for $g$ over $L$. In particular, this is a
    splitting field for $fg$ over $k$. We wish to show that $S = L$.

    Consider another root $b$ of $g\in S$. We have the diagram, here in two
    parts because I can't live-TeX properly:

    $$ k \subseteq k(a) \subseteq L \subseteq L(b) \subseteq S$$

    $$ k \subseteq k(b) \subseteq L(b) \subseteq S $$

    Notice that $k(a) \cong k(b)$. In fact, this is a $k$-isomorphism. By the
    ``fundamental lemma'', the isom extends the obvious map that sends $a$ to $b$.

    The degree $[S:k(a)]$ equals $[S:k(b)]$. Now, by the tower law,
    \begin{align*}
      [L(b):k(a)] &= [L(b):k(b)] \\
      [L(b):L][L:k(a)] &= [L(b):k(b)]
    \end{align*}

    $L(b)$ is a splitting field of $f$ over $k(b)$, and $L$ is a splitting field
    of $f$ over $k(a)$. Using the fundamental lemma again,
    $$[L:k(a)] = [L(b):k(b)]$$
    which implies that $[L(b):L]=1$, so that $b \in L$.
  \end{enumerate}
\end{proof}

\begin{defn}
 An extension of fields is \textit{Galois} if it is normal and separable.
\end{defn}

\begin{thm}[Fundamental theorem of Galois theory]
  If $L/K$ is a finite Galois extension, then
  \begin{enumerate}
  \item There is a bijective correspondence

    $$\{\text{subgroups of } G_{L/K}\} \longleftrightarrow \{\text{intermediate fields } L/M/K\}$$
  \item $G_{L/M}$ is normal in $G_{L/K}$ iff $M$ is normal in $K$.
  \end{enumerate}
\end{thm}


Suppose $L/M/K$ is a tower of normal extensions, so $G_{L/M}$ is normal in
$G_{L/K}$. What is the quotient $G_{L/K}/G_{L/M}$?

We can restrict $\sigma\in G_{L/K}$ to $M$ to get a $K$-aut of $M$. We need that
$m\in M \implies \sigma(m)\in M$, but this holds because the minimal polynomial
of $\sigma(m)$ is the same as that of $m$ (fundamental lemma again). We get that

$$G_{L/K}/G_{L/M} \cong G_{M/K}$$

\begin{proof}
  \begin{enumerate}
  \item Start with $H$ and take $F_H$ and $G = G_{L/F_H}$. We wish to show that $G=H$.
First, $H\subseteq G$ ``by logic''. The other direction is simply that $|G| \leq
[L:F_H] = |H|$, and we're done.
\item Compare $[L:M]$ and $[L:F_{G_{L/M}}] = |G_{L/M}|$. We need to show that
  these two are equal. First, normality of $L/K$ implies $L/M$. The same goes
  for separability. 

  So we need to show that $L$ normal and separable implies that there are
  exactly $[L:M]$ of $L$. We will use induction.
  \begin{enumerate}
  \item The basis step $M/M$ is trivial. 
  \item Take any $a\in L$, with $a\not\in K$. If $m_a(x)$ is the minimal
    polynomial of $a$ over $K$, with $\deg m_a=d$, then $m_a$ has $d$ distinct
    (because separable) roots in $L$ (because normal).

    Say the roots are $a = a_1, a_2, \ldots, a_d$. We have
    $$K\subseteq K(a) \subseteq L$$
    where the first extension is degree $d$ and the second one is $[L:K]/d$.

    By the induction hypothesis, we have $[L:K]/d$ $K$-automorphisms of $L$ that
    fix $a$. We now want to show that there are, for each $i$, exactly $[L:K]/d$
    $K$-automorphisms that send $a$ to $a_i$.

    We use the standard isomorphism
    $$\varphi_i: K(a) \cong K[x]/\bkt{m_a(x)} \cong K(a_i)$$

    Now $L$ is a splitting field (normality) of some $f\in K[x]$, so it is the splitting
    field of that same $f$ over both $K(a)$ and $K(a_i)$. Using the fundamental
    lemma, there is a $K$-automorphism of $L$ extending $\varphi_i$ sending $a$
    to $a_i$.

    The trick now is to compose this $K$-automorphism, call it $\sigma$, with
    the $s = [L:K]/d$ $K$-automorphisms of $L$ that fix $a$, to get
    $$\sigma\circ\sigma_1,\ldots,\sigma\circ\sigma_s$$
    which are exactly $K$-automorphisms of $L$ that send $a$ to $a_i$. In all,
    we have $sd = [L:K]$ $K$-automorphisms of $L$.
  \end{enumerate}
\end{enumerate}
\end{proof}

\section{Toward unsolvability of the quintic}
Given a polynomial $f\in\Q[x]$, one can determine whether or not the roots of
$f$ can be computed using radicals, starting from the coefficients, by looking
at $\gal L \Q$. In particular, we have the following:
\begin{thm}
  $f$ is solvable by radicals $\iff$ $\gal L \Q$ is ``solvable''.
\end{thm}

\subsection{Solvable groups}
For a group $G$, the \textit{commutator} of $a,b\in G$ is the element $[a,b]:= ab\iv a\iv
b$. The \textit{commutator subgroup} $G' = [G,G]$ is the subgroup generated by
all the commutators.

It is easily checked that $G/G'$ is abelian. Furthermore, the commutator
subgroup satisfies a kind of universal property:

\begin{thm}
 $G/N$ is abelian $\iff$ $N\supseteq [G,G]$. 
\end{thm}

Now we define solvable groups:
\begin{thm}(-definition)
  TFAE:
  \begin{enumerate}
  \item The chain of subgroups
    $$G \supseteq G' \supseteq G''\supseteq\cdots$$
    hits $1$ after a finite number of steps.
  \item There is a chain of normal subgroups (\textit{composition series}?) of
    $G$:
    $$G=N_0 \supseteq N_1 \supseteq \cdots \supseteq N_k = 1$$
    where $N_i/N_{i+1}$ is abelian for all $i$.
  
  \item There is a chain as above, where each $N_{i+1}$ is normal in $N_i$ (but not
    necessarily in $G$). 
  \end{enumerate}
\end{thm}
\begin{thm}
 There are polynomials $f\in\Q[x]$ such that the Galois group of $f$ ($=\gal
 {L} \Q$ with $L = $ the splitting field of $f$ over $\Q$) is $S_5$. 
\end{thm}

Here is an explicit example:

$$f(x) = 2x^5 - 10x + 5$$
How do we know?

\begin{enumerate}
\item $f(x)$ is $5$-Eisenstein and is hence irreducible over $\Q$. Also, look
  mod $5$ to notice that both (purported) polynomial factors of $f$ must have a
  constant term of $0$ mod $5$, which would make the constant term of $f$
  divisible by $25$, which it is not. (I have a feeling that this is basically
  Eisenstein.)
\item So the splititng field has degree divisible by $5$, by the tower law. This
  tells us that the order of the Galois group $\gal L \Q$ is divisible by $5$.

  \begin{thm}[Cauchy]
    If $G$ is a finite group with $p$ dividing $|G|$, then $G$ has an order-$p$
    element.
  \end{thm}

  This tells us that $\gal L \Q$ has an order-$5$ element. This must necessarily
  be a $5$-cycle.

  Now a bit of graphing gives us that $f$ has three real roots, and a pair of
  complex conjugates. So $z\mapsto\bar z$ is a transposition in the Galois
  group. We can now show that these two elements (the transposition and the
  $5$-cycle) generate $S_5$.
  
\end{enumerate}

\section{Cubics}

\begin{thm}[Roots of a cubic]
  The reduced cubic
  $$f(x) = x^3 = ax = b$$
  has roots
  $$u_1+u_2,\zeta u_1+\zeta^2 u_2, \zeta^2 u_1+\zeta u_2$$
  where
  $$u_{12} = \sqrt[3]{\frac b2 \pm \sqrt D}$$
  $$D = \frac {b^2}4 - \frac{a^3}{27}$$
\end{thm}

Note that to express the roots of $f$ as radicals, you may have to go beyond a
splitting field of $f$. (Why?)

\begin{defn}
 $f\in\Q[x]$ is solvable by radicals if its splitting field (or, equivalently,
 one of its roots) is contained in some [``repeated''] \textit{radical extension} of $\Q$. 
\end{defn}

\begin{defn}
  A radical extension is a field $F$ obtained from $\Q$ by adjoining an $n$-th
  root at each step of a finite tower.
\end{defn}

\begin{thm}
 If (and only if) $f\in\Q[x]$ is solvable by radicals, the Galois group of the
 splitting field $L$, $\gal L\Q$, is a solvable group.

 In particular, if $\gal L\Q$ is $S_n, n\geq 5$, then $f$ is not solvable by radicals.
\end{thm}
\begin{proof}[Sketch of proof]
  We have the radical extension
  $$\Q\subseteq K\subseteq \Q[a_1,a_2,\ldots,a_r]$$
  which we will extend to a Galois (and still radical) extension of $\Q$. The
  idea is to take the minimal polynomials of the $a_i$ and use the splitting
  field of the product $F$ of these. Call this (normal closure?) $L*$.

  We have the tower
  $$\Q\subseteq L\subseteq L^*$$
  where $\gal {L^*} L$ is normal, and $\gal {L^*} \Q / \gal {L^*} L \cong \gal L
  \Q$. Since quotients of solvable groups are solvable, if we can show that
  $\gal {L^*} \Q$ is solvable, we are done.

  To start, we adjoin all $n$-th roots of unity that we might need at the
  beginning. For instance, we use the tower
  $$\Q \subseteq \Q(i)\subseteq\Q(\sqrt[4]2,i)$$
  instead of the other one to factor $x^4-2$.

  \begin{prop}
    If $K$ contains all $n$-th roots of unity and $L=K(a)$ where $a^n \in K$,
    then $\gal L K$ is abelian.
  \end{prop}
  \begin{proof}
  $\sigma \in \gal L K$ is determined by $\sigma(a)$, which must be another root
  of $x^n - c$.

  So $\sigma : a\mapsto \zeta^k a$ for some $k$. Note that $\zeta \in K$ by
  assumption. Any two such $\sigma$ will commute, since they are just
  multiplication maps.
 \end{proof}
 \begin{prop}
   $\gal {\Q(\zeta_n)} \Q$ is abelian.
 \end{prop} 
 \begin{thm}
  Obvious. 
 \end{thm}
 With this, the tower of fields
 $$\Q \subseteq \Q(a_1) \subseteq \cdots \subseteq \Q(a_1,\ldots,a_r)$$
 is normal at each step, where the first extension is for the roots of unity
 (which is okay by the second prop.) and the others are just adjunctions of
 other roots, which our first proposition takes care of.
\end{proof}

\end{document}
\documentclass{article}

\input{/home/mrkgnao/math/preamble.tex}

\title{Elliptic curve factoring}
\author{David, transcribed by Soham}
\date{July 2016}

\begin{document}
\maketitle
 
Let $N$ be a large (but not huge) number. Goal: find the factorization of $N$
into primes on a classical computer.

\begin{enumerate}
\item Bogofactor.
\item Simulate Shor's algorithm.
\item Trial division upto $n-1$.
\item Trial division upto $\sqrt n$.
\item Fermat method: try writing $N=x^2-y^2$ nontrivially.
\item Generalize Fermat to $p$-th powers.
\item Generalize Fermat by looking at $x,y$ with
  $$x^2-y^2 \equiv 0 \Mod n $$

\item Take $x>\sqrt N$. If $c$ only has small prime factors, we can assemble a
  ``difference of squares'' factorization from such $c$. As an example, consider (?) 
\end{enumerate}

\section{Pollard $p-1$ algorithm}
$N = pq$. Suppose that $p-1$ is $B$-smooth, but $q-1$ is not. For any
$a\not\equiv 0\Mod p$ or $\Mod q$, define
$$M=\prod_{p\leq B}p^{\lfloor \log_p B \rfloor} = \lcm{(1,\ldots,B)}$$

Now choose $a$ randomly and compute $a^M\equiv 1\Mod p$. Usually, we also have
$a^M\not\equiv 1\Mod q$. Then $\gcd(a^M-1,N)$ will often be a factor of $N$.

\section{Day 2}
\subsection{Elliptic curves}
A set of points in the plane defines an elliptic curve if $y^2 = x^3+ax^2+bx+c$.
This is called Weierstrass form, and it exists in characteristic not equal to
$2$. In characteristic not $3$, we can change variables further to get $y^2 = x^3+ax+b$.

How many points does an arbitrary line intersect an elliptic curve $E$ in?
Morally, of course, it should be $3$, by Bézout's theorem. Nic fixed this in his
colloquium by:
\begin{itemize}
\item working over $\C$
\item using projective coordinates
\item counting points with multiplicity
\end{itemize}

We will, instead, prefer to use a group law on the elliptic curve to fix the
first one, because we want to work over finite fields.

Claim: If a line intersects $E$ in two points (with multiplicity), then it
intesects in a third point.

This is only true if we are working in some projective space, though. Suppose
$y^2 = x^3+ax^2+bx+c$. Given $P_1=(x_1,y_1)$ and $P_2 = (x_2,y_2)$, consider the
line $y=\lambda x+\nu$ through them.

We get, after using the formula $x_1+x_2+x_3 = \lambda^2 - a$, the expression

$$x_3 = \lambda^2 - a - x_1 - x_2$$

where $$\lambda = \frac{y_2-y_1}{x_2-x_1}.$$

This breaks down for $x_2=x_1$. The line we use in such cases is:
\begin{itemize}
\item If $P_1=P_2$, we just use the tangent line:
  $$\lambda = \frac{3x_1^2+2a+b}{2y_1}.$$
\item For ``vertical'' lines, we use a point $\mathcal O$ at infinity.
\end{itemize}

\subsection{Projectivization}
Consider $y^2z = x^3+ax^2z+bxz^2+cz^3$. This is homegeneous upto rescaling, and
solutions $[x:y:z]$ come up to rescaling. When $z\neq 0$, we can rescale to
$[x:y:1]$. When $[z=0]$, $x^3 = 0 = x$ so we can rescale to $[0:1:0]$, which we
will use as the identity for the group law.

\subsection{The group law}
Let $P\ast Q$ be the third point of intersection of the line through $P$ and $Q$
with $E$. Then the reflection of $P\ast Q$ across the $x$-axis is called $P+Q$.

\begin{thm}
 This gives us an abelian group structure on the points of the elliptic curve
 (including $\mathcal O$).
\end{thm}

Associativity can be brute-forced, or one can look at the (moduli?) space of
elliptic curves and use dimension arguments arising from the different
expressions

$\mathcal O, P, Q, R, P\ast Q, P\ast R, Q\ast R, (P+Q)\ast R, P\ast(Q+R)$
$C_1,C_2,C_3$ cubics, $C_1,C_2$ intersect in $9$ points, if $C_3$ passes through
$8$ of those points, then it passes through the $9$th one too.

\section{Day 2}
\begin{question}
 If $E$ is an elliptic curve, how many elements does $E(\F_p)$ have? 
\end{question}
$y^2=x^3+bx+c$. Heuristically, as $x$ varies, $x^3+bx+c$ is a square in $\F_p$ half of the
time, so we expect $p+1$ points.

\begin{thm}[Hasse]
   $|p+1-\#E({\F_p})| \leq 2\sqrt p$.
\end{thm}

\begin{thm}[Sato-Tate]
 The distribution of the number of points on non-CM elliptic curves varies with
 a probability distribution that looks like a semicircle.
\end{thm}

\begin{defn}
  $$L_n[a,c] := O(e^{(c+o(1)){(\log n)}^a{(\log \log n)}^{1-a}}).$$
\end{defn}

In particular, for $a=0$, we get ${(\log n)}^{c+o(1)}$, and for $a=1$, we get $n^{c+o(1)}$.

\section{Day 3}
Given a curve $E$, find $\#E(\F_p)$.

\begin{enumerate}
\item For each $x$, determine whether $x^3+ax+b$ is a square mod $p$. Each
  nonzero square contributes $2$ points.
\item Pick a random point $P$ on the curve and compute $P,2P,\cdots$ until you
  reach $\mathcal O$.
\item ``Baby step --- giant step''. Pick a point $P$ and compute
  $$P,2P,3P,\cdots.$$
  Let $Q=-BP$. Now compute
  $$Q,2Q,3Q,\cdots.$$
  There is a point that is on both lists. We get that
  $$iP = -jQ = -jBP,$$
  so $$(i+Bj)$$ is a multiple of the order of the group.
\item Let $a=p+1-\#E(\F_p)$.
  We will find $a\Mod l$ for lots of small primes $l$. Once $\pi l>4\sqrt l$, we
  can use the Chinese remainder theorem to recover $a$.

  Warmup: Find $a\Mod 2$.
  $2$ divides $\#E(\F_p)\iff x^3+Ax+B$ has a root $\alpha$ in $\F_p$. We know
  that any such $\alpha$ has to satisfy $x^p-x$ as well.

  So a root exists iff $\gcd(x^p-x, x^3+Ax+B) = \gcd(x^p\Mod {x^3+Ax+B} - x,
  x^3+Ax+B) \neq 1$. Modular exponentiation is fast!
\end{enumerate}

Let us generalize to $p\neq 2$.
\begin{defn}
  The \textit{$n$-torsion} $E[n]$ of $E$ is defined as
  $$E\supset E[n] := \{P\in E(\overline{\F_p}):nP = \mathcal O\}.$$
\end{defn}

We can try to find $a\Mod l$ by studying $E[l]$. In particular, if $E[l](\F_p)
\neq 0$ then $l|\#E(\F_p)$.

\end{document}
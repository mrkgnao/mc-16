\documentclass{article}

\input{/home/mrkgnao/math/preamble.tex}

\title{The Yoneda lemma}
\author{Sachi, transcribed by Soham}
\date{August 2016}
\begin{document}
\maketitle


\begin{itemize}
\item Covariant hom-functor: $\hhom a$, with the action on morphisms given by
  maps $\homf a x\to\homf a y$, $$f\mapsto (-\circ f).$$
\item Contravariant hom-functor: $\homm a$, with the action on morphisms given by
  maps $\homf a x\leftarrow\homf a y$, $$f\mapsto (f\circ -).$$
\end{itemize}

\begin{defn}
 We say that a functor $F: C\to\Set$ is \textit{representable} if there exists a
 natural transformation such that
 $$\alpha: F\simeq\hhom a.$$
\end{defn}

Examples:
\begin{itemize}
\item Forgetful functors, e.g. $F: \Grp\to\Set$. Notice that $\homf \Z G$ is
  completely determined by where $1$ goes, so it is sort of like the set of maps
  from a one-point test space. So elements of $\homf\Z G$ correspond to elements
  of $G$.
\end{itemize}

All of this is trivial, of course.

\end{document}